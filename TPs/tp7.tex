\subsection{Question 1}
    \textbf{Translate the following function to IR in SSA form and determine the liveness ranges of the variables. Draw the interference graph. Then, allocate registers for an ARM-like CPU and generate machine
code, assuming that the parameter “c” of the function is passed in register r1 (without using the stack. That’s more efficient!) and that the return value of the function should be in register r0.}

    
\begin{table}[H]
    \centering
    \begin{tabular}{|c|c|c|c|c|c|c|c|}
         \hline & t1 & c1 & a1 & t0 & a0 & b0 & c0 \\
         \hline a0 = c0 $\cdot$ 2 &  &  &  &  &  &  & live\\
         \hline b0 = a0 + 1 &  &  &  &  & live &  & live\\
         \hline c1 = c0 + b0 &  &  &  &  & live & live & live \\
         \hline t0 = b0 $\cdot$ 2&  & live &  &  & live & live & \\
         \hline a1 = t0 + a0 &  & live &  & live & live &  & \\
         \hline t1 = c1 + a1 &  & live & live &  &  &  & \\
         \hline return t1& live &  &  &  &  &  & \\
         \hline
    \end{tabular}
    \label{tab:my_label}
\end{table}

    \addimg{img/TP7/1_b.pdf}{}{Interference graph}{}

    \begin{lstlisting}[language = Java , frame = trBL , firstnumber = last , escapeinside={(*@}{@*)}]
pop r0
c0 = r1 (required by the ABI)
a0 = r0
b0 = r2
c1 = r1
t0 = r2
a1 = r0
t1 = r0 (required by the ABI)
\end{lstlisting}


\subsection{Question 2}
    \textbf{In the course, we only saw code generation for functions with parameters and local variables. Now,
let’s look at an example with global variables:}

    \begin{lstlisting}[language = Java , frame = trBL , firstnumber = last , escapeinside={(*@}{@*)}]
int x;
int y[10];

void add(int v) {
    y[x] = v;
    x++;
}
\end{lstlisting}


    \textbf{Translate the above function “add” first to IR code and then to machine code. Assume that the global
variables “x” and “y” start at address 0x10000 and 0x10004, respectively, in main memory.}
    \begin{lstlisting}[language = Java , frame = trBL , firstnumber = last , escapeinside={(*@}{@*)}]
// parameter is v0
t0 = 0x10004
t1 = *t0
 // read value of x
t2 = t1 * 4
 // each array element is four bytes
t3 = 0x10000
t4 = t0 + t3
 // address of y[x]
*t4 = v0
 // store v in y[x]
//-----------
t5 = 0x10004
t6 = *t5
t7 = t6 + 1
*t5 = t7
 // store x+1 in x
\end{lstlisting}

Of course, this code can be optimized. Instead of loading again the variable x in t6, we could just re-use t1.

\subsection{Question 3}
    \textbf{Function calls are expensive because they involve a lot of operations (pushing the arguments on the stack, making backups of registers, jumping to the function, etc.). Many compilers can perform an optimization called function inlining where the code of the called function is directly inserted at the call location, thus avoiding the call. Let’s look at the following example:}

    \begin{lstlisting}[language = Java , frame = trBL , firstnumber = last , escapeinside={(*@}{@*)}]
int f(int v) {
    v = v + 2;
    return v*v;
}
int g(int a, int b) {
    int c = f(a);
    int d = 2*f(a+b);
    return c+d;
}
\end{lstlisting}


    \textbf{Take the role of the compiler and inline the function f at the two places where it is called. Do this inthe IR, not in the source code. Think about a strategy how to handle the variables. By the way, compilers only inline small functions. What could be the reason?}

    \begin{lstlisting}[language = Java , frame = trBL , firstnumber = last , escapeinside={(*@}{@*)}]
// parameters are a0 and b0
v0 = a0
 // calling function f with argument a
v1 = v0 + 2
t0 = v1 * v1
c0 = t0
 // c = f(a)
//-----------
v2 = a0 + b0
 // calling function f with argument a+b
v3 = v2 + 2
t1 = v3 * v3
d0 = 2 * t1
 // d = 2*f(a+b)
//-----------
t2 = c0 + d0
return t2
\end{lstlisting}

Why do compilers only inline small functions? Code size! Inlining increases the number of
instructions in the code and therefore reduces the efficiency of the instruction cache.

\subsection{Question 4}
    \textbf{In most CPUs, integer multiplications are slower than additions or bit shifting. Think about ways to reduce the strength of multiplications, i.e., find ways to avoid the multiplication instruction for arithmetic expressions where one of the operands is a constant, for example }

    \begin{center}
    $x\cdot2$ \\
    $x\cdot3$ \\
    $x\cdot4$ \\
    $x\cdot5$ \\
\end{center}    

    \begin{center}
    $x\cdot2$ = x+x or  x$\verb|<<|$1 \\
    $x\cdot3$ = x+x+x or  x$\verb|<<|$1 + x \\
    $x\cdot4$ = x$\verb|<<|$2 \\
    $x\cdot5$ = x$\verb|<<|$2 + x \\
    x  $\cdot(2^n + 1)$ = (x$\verb|<<|$n) + x
\end{center}