$1 (01|1)^{*}00(1|10)^{*}$