\begin{lstlisting}[language = Java , frame = trBL , firstnumber = last , escapeinside={(*@}{@*)}]
// parameters are a0 and b0
v0 = a0
 // calling function f with argument a
v1 = v0 + 2
t0 = v1 * v1
c0 = t0
 // c = f(a)
//-----------
v2 = a0 + b0
 // calling function f with argument a+b
v3 = v2 + 2
t1 = v3 * v3
d0 = 2 * t1
 // d = 2*f(a+b)
//-----------
t2 = c0 + d0
return t2
\end{lstlisting}

Why do compilers only inline small functions? Code size! Inlining increases the number of
instructions in the code and therefore reduces the efficiency of the instruction cache.